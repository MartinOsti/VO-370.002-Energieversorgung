\documentclass[11pt,a4paper]{scrartcl}
\usepackage{a4wide}
\usepackage{fancyhdr}
\usepackage[naustrian]{babel}
\usepackage[utf8]{inputenc}
\usepackage{enumerate}%Aufzählungen
\usepackage{amsmath}%Formeln
\usepackage[locale=DE]{siunitx}%Einheiten
\usepackage{eurosym}%Eurosymbol
\usepackage{tikz}%Zeichnungen
\usepackage{pgfplots}%Funktionen plotten
\usepackage[european]{circuitikz}%Schaltungen
\usetikzlibrary{decorations.pathreplacing}


\pagestyle{fancy} %eigener Seitenstil
\fancyhf{} %alle Kopf- und Fußzeilenfelder bereinigen
\fancyhead[L]{Energieversorgung\\VO 370.002} %Kopfzeile links
\fancyhead[C]{Lösung zur Prüfung vom 29.04.2014} %zentrierte Kopfzeile
\fancyhead[R]{www.fet.at} %Kopfzeile rechts
\renewcommand{\headrulewidth}{0.4pt} %obere Trennlinie
\fancyfoot[C]{\thepage} %Seitennummer
\renewcommand{\footrulewidth}{0.4pt} %untere Trennlinie


%%%%%%% Makro für Schalter in CircuiTikz %%%%%%%
% modified code from pgfcircbipoles.sty and circuitikz1.code.tex
\makeatletter
% create the shape
\pgfcircdeclarebipole{}{\ctikzvalof{bipoles/interr/height 2}}{spst}{\ctikzvalof{bipoles/interr/height}}{\ctikzvalof{bipoles/interr/width}}{

    \pgfsetlinewidth{\pgfkeysvalueof{/tikz/circuitikz/bipoles/thickness}\pgfstartlinewidth}

    \pgfpathmoveto{\pgfpoint{\pgf@circ@res@left}{0pt}}
    \pgfpathlineto{\pgfpoint{.6\pgf@circ@res@right}{\pgf@circ@res@up}}
    \pgfusepath{draw}   
}
% make the shape accessible with nice syntax
\def\pgf@circ@spst@path#1{\pgf@circ@bipole@path{spst}{#1}}
\tikzset{switch/.style = {\circuitikzbasekey, /tikz/to path=\pgf@circ@spst@path, l=#1}}
\tikzset{spst/.style = {switch = #1}}
\makeatother
%%%%%%%%%%%%%%%%%%%%%%%%%%%%%%%%%%%%%%%%%%%%%%%


\DeclareSIUnit\year{a}
\DeclareSIUnit\wattelectrical{W_{el}}
\DeclareSIUnit\hourelectrical{h_{el}}
\DeclareSIUnit{\EUR}{\text{\euro}}
\sisetup{
  per-mode = fraction,
}

\newcommand{\mybr}[1]{\left(#1\right)}
\newcommand{\ugamma}{\underline{\gamma}}
\renewcommand{\j}{\mathrm{j}}
\newcommand{\Z}{\underline{Z}}
\renewcommand{\S}{\underline{S}}
\newcommand{\U}{\underline{U}}
\newcommand{\I}{\underline{I}}
\newcommand{\E}{\underline{E}}
\newcommand{\0}{_{\mybr{0}}}
\newcommand{\1}{_{\mybr{1}}}
\newcommand{\2}{_{\mybr{2}}}
\renewcommand{\a}{\underline{a}}


\begin{document}
\section{Leitungsgleichungen}
\begin{enumerate}[a)]
\item
\begin{align}
\ugamma&=\sqrt{\mybr{R'+\j\omega L'}\mybr{G'+\j\omega C'}}\\
\Z_W&=\sqrt{\frac{R'+\j\omega L'}{G'+\j\omega C'}}\\
P_{nat}&=\frac{U_n^2}{Z_W}
\end{align}
\item
\begin{equation}
R_L=\frac{U_n^2}{P_L}
\end{equation}
\item
\begin{equation}
\S_1=\U_1\I_1^*=\U_1\frac{\U_1^*}{\Z_W^*}=\frac{U_n^2}{\Z_W^*}
\end{equation}
\item
\begin{align}
\U_1&=\cosh\mybr{\ugamma l}\U_2+\sinh\mybr{\ugamma l}\Z_W\I_2\\
&=\cosh\mybr{\ugamma l}\U_2+\sinh\mybr{\ugamma l}\Z_W\frac{\U_2}{R_L}\\
\U_2&=\frac{\U_1}{\cosh\mybr{\ugamma l}+\sinh\mybr{\ugamma l}\frac{\Z_W}{R_L}}
\end{align}
\item
\begin{align}
\U_1&=\cosh\mybr{\ugamma l}\U_2+\sinh\mybr{\ugamma l}\Z_W\I_2\\
&=\cosh\mybr{\ugamma l}\SI{0,9}\U_1+\sinh\mybr{\ugamma l}\Z_W\frac{\SI{0,9}\U_1}{\Z_2}\\
\sinh\mybr{\ugamma l}\Z_W\frac{\SI{0,9}\U_1}{\Z_2}&=\U_1-\cosh\mybr{\ugamma l}\SI{0,9}\U_1\\
\Z_2&=\frac{\sinh\mybr{\ugamma l}\Z_W\SI{0,9}\U_1}{\U_1-\cosh\mybr{\ugamma l}\SI{0,9}\U_1}
\end{align}
\item
\end{enumerate}
\section{Ein- und zweipoliger Kurzschluss}
\begin{enumerate}[a)]
\item

\begin{figure*}[!h]
\centering
\begin{circuitikz}
\begin{scope}[scale=0.8]
	%\draw [help lines] (-1,-1) grid (12,17); %Zeichnet Raster und vereinfacht damit das Zeichnen
	
	%Nullsystem
	\draw (0,0) -- (0,3)
	to[switch] (2,3)
	to[R, l=$X_T$] (4,3)
	to[R, l=$X_0$] (6,3)
	-- (7,3);
	\draw (0,0) -- (7,0);
	\draw (4,3) to[C, l=$C\0$, *-*] (4,0);
	\draw node[ocirc] (B0) at (7,3) {}	node[ocirc] (A0) at (7,0) {} (A0) to [open, v=$\U\0$] (B0);
	\draw (7,3) to[short, i>^=$\I\0$] (8,3);

	%Gegensystem
	\draw (0,6) -- (0,9)
	to[R, l=$X_d''$] (2,9)
	to[R, l=$X_T$] (4,9)
	to[R, l=$X_L$] (6,9)
	-- (7,9);
	\draw (0,6) -- (7,6);
	\draw node[ocirc] (B2) at (7,9) {}	node[ocirc] (A2) at (7,6) {} (A2) to [open, v=$\U\2$] (B2);
	\draw (7,9) to[short, i>^=$\I\2$] (8,9);
	
	% Mitsystem
	\draw (0,15) to[V, v^=$\E\1{=}c\frac{U_n}{\sqrt{3}}$] (0,12);
	\draw (0,15) to[R, l=$X_d''$] (2,15)
	to[R, l=$X_T$] (4,15)
	to[R, l=$X_L$] (6,15)
	-- (7,15);
	\draw (0,12) -- (7,12);
	\draw node[ocirc] (B1) at (7,15) {}	node[ocirc] (A1) at (7,12) {} (A1) to [open, v=$\U\1$] (B1);
	\draw (7,15) to[short, i>^=$\I\1$] (9,15);
	
	%Verbindung
	\draw (7,0) -| (9,15);
	\draw (8,3) |- (7,6);
	\draw (8,9) |- (7,12);
	
	%Kästchen
	\draw (-1,-1) -| (6.5,4.2) -| node[near end,left] {$\mybr{0}$} (-1,-1);
	\draw (-1,5) -| (6.5,10.2) -| node[near end,left] {$\mybr{2}$} (-1,5);
	\draw (-1,11) -| (6.5,16.2) -| node[near end,left] {$\mybr{1}$} (-1,11);
\end{scope}
\end{circuitikz}
\end{figure*}

\item
\begin{align}
X_d''&=x_d''\frac{U_2^2}{S_N}\\
X_T&=u_k\frac{U_2^2}{S_N}\\
X_L&=\omega L_B' l\\
C\0&=C_E' l\\
\I\1&=\frac{c\frac{U_n}{\sqrt{3}}}{2\j\mybr{X_d''+X_T+X_L}+\j X_0+\frac{1}{\j\omega C\0}}\\
\I_{k1p}''&=\I_a=\I\0+\I\1+\I\2=3\I\1
\end{align}
\item
\begin{align}
\j\omega 3L_p+\j X_T+\frac{1}{\j\omega C\0}&=0\\
L_p&=\frac{1}{3\omega^2 C\0}-\frac{X_T}{3\omega}
\end{align}
\item

\begin{figure*}[!h]
\centering
\begin{circuitikz}
\begin{scope}[scale=0.8]
	%\draw [help lines] (-1,-1) grid (12,17); %Zeichnet Raster und vereinfacht damit das Zeichnen
	
	%Nullsystem
	\draw (0,0) to[american inductor, l_=$3L_p$] (0,3)
	-- (2,3)
	to[R, l=$X_T$] (4,3)
	to[R, l=$X_0$] (6,3)
	-- (7,3);
	\draw (0,0) -- (7,0);
	\draw (4,3) to[C, l=$C\0$, *-*] (4,0);
	\draw node[ocirc] (B0) at (7,3) {}	node[ocirc] (A0) at (7,0) {} (A0) to [open, v=$\U\0$] (B0);
	\draw (7,3) to[short, i>^=$\I\0$] (9,3);

	%Gegensystem
	\draw (0,6) -- (0,9)
	to[R, l=$X_d''$] (2,9)
	to[R, l=$X_T$] (4,9)
	to[R, l=$X_L$] (6,9)
	-- (7,9);
	\draw (0,6) -- (7,6);
	\draw node[ocirc] (B2) at (7,9) {}	node[ocirc] (A2) at (7,6) {} (A2) to [open, v=$\U\2$] (B2);
	\draw (7,9) to[short, i>^=$\I\2$, -*] (9,9);
	
	% Mitsystem
	\draw (0,15) to[V, v^=$\E\1{=}c\frac{U_n}{\sqrt{3}}$] (0,12);
	\draw (0,15) to[R, l=$X_d''$] (2,15)
	to[R, l=$X_T$] (4,15)
	to[R, l=$X_L$] (6,15)
	-- (7,15);
	\draw (0,12) -- (7,12);
	\draw node[ocirc] (B1) at (7,15) {}	node[ocirc] (A1) at (7,12) {} (A1) to [open, v=$\U\1$] (B1);
	\draw (7,15) to[short, i>^=$\I\1$] (9,15);
	
	%Verbindung
	\draw (7,0) -| (10,12) -- (7,12);
	\draw (9,3) |- (9,15);
	\draw (7,6) to[short, -*] (10,6);
	
	%Kästchen
	\draw (-1,-1) -| (6.5,4.2) -| node[near end,left] {$\mybr{0}$} (-1,-1);
	\draw (-1,5) -| (6.5,10.2) -| node[near end,left] {$\mybr{2}$} (-1,5);
	\draw (-1,11) -| (6.5,16.2) -| node[near end,left] {$\mybr{1}$} (-1,11);
\end{scope}
\end{circuitikz}
\end{figure*}

\item
\begin{align}
\I\0&=\SI{0}{\ampere}\\
\I\1&=-\I\2=\frac{c\frac{U_n}{\sqrt{3}}}{2\j\mybr{X_d''+X_T+X_L}}\\
\I_b&=\I\0+\a^2\I\1+\a\I\2=\mybr{\a^2-\a}\I\1=-\j\sqrt{3}\I\1\\
\I_c&=\I\0+\a\I\1+\a^2\I\2=\mybr{\a-\a^2}\I\1=\j\sqrt{3}\I\1
\end{align}
\item
In der Skizze sind die wahren Flussrichtungen und nicht die Orientierungen eingezeichnet.
\begin{figure*}[!h]
\centering
\begin{circuitikz}
\begin{scope}[scale=0.8]
	%\draw [help lines] (-1,-1) grid (8,5); %Zeichnet Raster und vereinfacht damit das Zeichnen
	
	\draw (0,0) -- node[near end,left] {$B$} (0,4);
	\draw (7,0) -- node[near end,right] {$C$} (7,4);
	\draw (0,1) 
	to[short] (6,1)
	to[short, i>^=$\I_c$] (7,1);
	\draw (0,2) 
	to[short] (6,2)
	to[short, i>^=$\I_b$] (7,2);
	\draw (0,3) 
	to[short] (7,3);
	
\end{scope}
\end{circuitikz}
\end{figure*}
\end{enumerate}
\section{A Transformator}
\begin{enumerate}[a)]
\item
\begin{equation}
R_k=P_k\frac{U_2^2}{S_N^2}
\end{equation}
\item
\begin{align}
P_L&=G_L U_L^2\\
G_L&=\frac{P_L}{U_L^2}=\frac{P_L}{U_2^2}
\end{align}
\item
\begin{align}
Q_L&=-B_L U_L^2\\
S_L&=U_L I_L=U_N I_N\cdot\SI{0,25}{\percent}=S_N\cdot\SI{0,25}{\percent}\\
Q_L&=\sqrt{S_L^2-P_L^2}\\
B_L&=-\frac{Q_L}{U_L^2}
\end{align}
\item
\begin{equation}
Z_k=u_k\frac{U_2^2}{S_N}
\end{equation}
\end{enumerate}
\setcounter{section}{2}
\section{B Parallelschaltung von zwei Transformatoren}
\begin{enumerate}[a)]
\item
\begin{figure*}[!h]
\centering
\begin{circuitikz}
\begin{scope}[scale=0.8]
	%\draw [help lines] (-1,-1) grid (7,4); %Zeichnet Raster und vereinfacht damit das Zeichnen
	
	%Schaltung
	\draw (0,0) -- (6,0);
	\draw (0,3) to[R, l_=$Z_{kT1}$] (3,3)
	to[R, l_=$Z_{kT2}$] (6,3);
	
	%Spannungspfeile
	\draw node[ocirc] (B0) at (3,3) {}	node[ocirc] (A0) at (3,0) {} (A0) to [open, v=$\U_{US}'$] (B0);
	\draw node[ocirc] (B1) at (0,3) {}	node[ocirc] (A1) at (0,0) {} (A1) to [open, v^=$\U_{US1}$] (B1);
	\draw node[ocirc] (B2) at (6,3) {}	node[ocirc] (A2) at (6,0) {} (A2) to [open, v=$\U_{US2}$] (B2);
	{
	\ctikzset{voltage/distance from node=.1}
	\ctikzset{voltage/bump b=5.0}
	\ctikzset{voltage/european label distance=4.5}
	\draw (B2) to [open, v=$\Delta\U_{US}$] (B1);
	}
	
\end{scope}
\end{circuitikz}
\end{figure*}
\begin{align}
\Delta\U_{US}&=\U_{US1}-\U_{US2}=\U_{OS}\mybr{\frac{1}{\underline{\textit{ü}}_1}-\frac{1}{\underline{\textit{ü}}_2}}\\
\U_{US}'&=\U_{US1}-\frac{Z_{kT1}}{Z_{kT1}+Z_{kT2}}\Delta\U_{US}
\end{align}
\item
\begin{equation}
\I_k=\frac{\Delta\U_{US}}{Z_{kT1}+Z_{kT2}}
\end{equation}
\item
\end{enumerate}
\section{Fünf Sicherheitsregeln}
Siehe Skriptum S.IX
\section{Wirtschaftlichkeitsrechnung}
\begin{enumerate}[a)]
\item
\begin{align}
\alpha&=\frac{\mybr{q-1}q^n}{q^n-1}=\frac{\mybr{\SI{1,05}{}-1}\SI{1,05}{}^{25}}{\SI{1,05}{}^{25}-1}=\SI{0,07095}{\per\year}\\
T_m&=\frac{E}{P}=\frac{\SI{2100}{\giga\watt\hour\per\year}}{\SI{300}{\mega\wattelectrical}}=\SI{7000}{\hour\per\year}\\
k&=\frac{\alpha a+c}{T_m}+b+d\\
&=\frac{\SI{0,07095}{\per\year}\cdot\SI{590}{\EUR\per\kilo\wattelectrical}+\SI{87}{\EUR\per\kilo\wattelectrical\per\year}}{\SI{7000}{\hour\per\year}}+\frac{\SI{0,37}{\EUR\per\cubic\metre}\cdot\SI{3,6}{\mega\joule\per\kilo\watt\per\hour}}{\SI{0,59}{}\cdot\SI{30}{\mega\joule\per\cubic\metre}}+\SI{0,001}{\EUR\per\kilo\watt\per\hourelectrical}\\
&=\SI{0,09466}{\EUR\per\kilo\watt\per\hourelectrical}
\end{align}
\item
\begin{align}
\alpha&=\frac{\mybr{q-1}q^n}{q^n-1}=\frac{\mybr{\SI{1,04}{}-1}\SI{1,04}{}^{40}}{\SI{1,04}{}^{40}-1}=\SI{0,05052}{\per\year}\\
T_m&=\frac{E}{P}=\frac{\SI{1560}{\giga\watt\hour\per\year}}{\SI{300}{\mega\wattelectrical}}=\SI{5200}{\hour\per\year}\\
k&=\frac{\alpha a+c}{T_m}+b+d\\
&=\frac{\SI{0,05052}{\per\year}\cdot\SI{3100}{\EUR\per\kilo\wattelectrical}+\SI{78}{\EUR\per\kilo\wattelectrical\per\year}}{\SI{5200}{\hour\per\year}}\\
&=\SI{0,04512}{\EUR\per\kilo\watt\per\hourelectrical}
\end{align}
\item
\begin{align}
k_{GuD}&=\frac{\alpha_{LWK} a_{LWK}+c_{LWK}}{T_{m,LWK}}\\
T_{m,LWK}&=\frac{\alpha_{LWK} a_{LWK}+c_{LWK}}{k_{GuD}}=\frac{\SI{0,05052}{\per\year}\cdot\SI{3100}{\EUR\per\kilo\wattelectrical}+\SI{78}{\EUR\per\kilo\wattelectrical\per\year}}{\SI{0,09466}{\EUR\per\kilo\watt\per\hourelectrical}}=\SI{2478}{\hour\per\year}
\end{align}
\item
\begin{align}
k_{GuD}&=\frac{\SI{128,9}{\EUR\per\kilo\wattelectrical\per\year}}{T_m}+\SI{0,07625}{\EUR\per\kilo\watt\per\hourelectrical}\\
k_{LWK}&=\frac{\SI{234,6}{\EUR\per\kilo\wattelectrical\per\year}}{T_m}
\end{align}

\begin{figure*}[!h]
		\centering
		\begin{tikzpicture}
			\begin{axis}[width=0.5\linewidth, xlabel={$T_m$ in $\si{\hour}$}, ylabel={$k_{GuD}$, $k_{LWK}$ in \si{\EUR\per\kilo\watt\per\hourelectrical}}, grid, xmin=20, xmax=8760,samples=100, ymin=0, ymax=0.3, legend to name=k-legende, name=kplot]
				\addplot[domain=500:8760, blue] {128.9/x+0.07625};
				\addlegendentry{$k_{GuD}$}
				\addplot[domain=700:8760, red]  {234.6/x};
				\addlegendentry{$k_{LWK}$}
				\addplot[domain=0:8760, black, style=dashed] {0.07625};
				\draw[style=dashed] (axis cs:1386,\pgfkeysvalueof{/pgfplots/ymin}) -- (axis cs:1386,\pgfkeysvalueof{/pgfplots/ymax});
				\draw[color=black] (axis cs:1386,0.1693) circle (2pt);
			\end{axis}
			\node at (kplot.east) [anchor=south, xshift=-1.50cm, yshift=1cm]
			{\pgfplotslegendfromname{k-legende}};
		\end{tikzpicture}%
\end{figure*}
\item
\end{enumerate}
\end{document}